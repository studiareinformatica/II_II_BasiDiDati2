\section{Un linguaggio diagrammatico}
Formato da costrutti, sintassi, semantica.
Descrive la struttura dei dati (\textbf{aspetto intensionale}) dei dati, non i dati stessi (\textbf{aspetto estensionale}).
Esempio: nella programmazione object-oriented, i campi di una classe descrivono l'aspetto intensionale dei dati - la costruzione di un'istanza della classe stessa deve descrivere il loro aspetto estensionale.

\subsection{Entità}
Rappresenta una classe di oggetti.
Le sue istanze hanno \textit{proprietà comuni} ed \textit{esistenza autonoma}. Ognuna avrà un proprio valore in un dato \textbf{dominio} (o \textit{tipo}). \\
\paragraph{Proprietà locale:} Ogni attributo di una classe ha valore solo su una specifica istanza.

\subsection{Domini}
Due tipi di domini visti finora:
\begin{itemize}
	\item \textbf{Domini specializzati} \hfill \\ I vincoli di dominio restringono insiemi di valori.\\
	Es: $int > 0; reale < 0; ...$
	\item \textbf{Domini enumerativi} \hfill \\ Definiscono esplicitamente un insieme finito di elementi.\\
	Es: sesso\{uomo, donna\}
\end{itemize}